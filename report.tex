\documentclass[12pt,letterpaper]{article}
\usepackage{fullpage}
\usepackage[top=2cm, bottom=4.5cm, left=2cm, right=2cm]{geometry}
\usepackage{amsmath,amsthm,amsfonts,amssymb,amscd}
\usepackage{lastpage}
\usepackage{enumerate}
\usepackage{enumitem}
\usepackage{fancyhdr}
\usepackage{mathrsfs}
\usepackage{xcolor}
\usepackage{graphicx}
\graphicspath{{images/}}
\usepackage{subcaption}
\usepackage{amsmath}
\usepackage{listings}
\usepackage{hyperref}
\usepackage{float}
\usepackage{empheq}
\usepackage{blindtext}
\usepackage[makeroom]{cancel}
%\usepackage[framed,numbered,autolinebreaks,useliterate]{mcode}
\documentclass{article}% or something else
\usepackage{pdfpages}
\usepackage{subfiles} % Best loaded last in preamble

\hypersetup{%
  colorlinks=true,
  linkcolor=blue,
  linkbordercolor={0 0 1}
}
 
\renewcommand\lstlistingname{Algorithm}
\renewcommand\lstlistlistingname{Algorithms}
\def\lstlistingautorefname{Alg.}

\lstdefinestyle{Python}{
    language        = Python,
    frame           = lines, 
    basicstyle      = \footnotesize,
    keywordstyle    = \color{blue},
    stringstyle     = \color{green},
    commentstyle    = \color{red}\ttfamily
}

\setlength{\parindent}{0.0in}
\setlength{\parskip}{0.05in}

% Edit these as appropriate
\newcommand\course{ME579}
\newcommand\hwnumber{HW 3}                  % <-- homework number
\newcommand\NetIDa{Colin Acton, Win Khoo, Rishi Jha}           % <-- NetID of person #1
\newcommand\duedate{}


% Common commands
\newcommand{\Laplace}[1]{\mathcal{L}\{#1\}}
\newcommand{\LaplaceInv}[1]{\mathcal{L}^{-1}\{#1\}}
\newcommand{\Z}[1]{\mathcal{Z}\{#1\}}
\newcommand{\ZInv}[1]{\mathcal{Z}^{-1}\{#1\}}
\newcommand\goesto{\rightarrow}
\newcommand\xvec{\textbf{x}}
\newcommand\dxvec{\dot{\textbf{x}}}
\newcommand\ddxvec{\ddot{\textbf{x}}}
\newcommand\tvec{\textbf{t}}
\newcommand\A{\textbf{A}}
\newcommand\B{\textbf{B}}
\newcommand\C{\textbf{C}}
\newcommand\D{\textbf{D}}
\newcommand\I{\textbf{I}}
\newcommand\T{\textbf{T}}
\newcommand\LambdaMat{\mathbf{\Lambda}}
\newcommand\Adj{\text{Adj}}
\newcommand\N{\textbf{N}}
\newcommand\J{\textbf{J}}
\newcommand\Q{\textbf{Q}}

\pagestyle{fancyplain}
\headheight 35pt
\lhead{\NetIDa}
\chead{\textbf{\Large Homework 3}}
\rhead{\course \\ \duedate}
\lfoot{}
\cfoot{}
\rfoot{\small\thepage}
\headsep 1.5em

\begin{document}

\section{Related Work}
Briefly summarize some approaches that have been tried in the literature [1]. This does not have to be comprehensive but is intended to help you think of different strategies. Since training your models can be quite time-consuming. We encourage you to think through different approaches before settling on your final algorithm.

\section{Cost Function}
Describe what you used as a cost function and why.

\section{Rewards}
Provide concrete definitions of your rewards, and include some motivation for why you picked them.

\section{State Representation}
Describe how you represent the state. If your algorithm computes features from states, clearly define them and provide motivation for why you picked them.  

\section{Approach}
Clearly describe your approach and the algorithm you used. Note that you do not necessarily have to use RL to solve this problem (e.g. you can use a variant of the cross-entropy method.)

\section{Evaluation}
Describe the performance of your agent, and make sure to include maximum and average lines cleared over 20 games.

\end{document}